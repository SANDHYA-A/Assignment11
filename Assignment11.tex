\documentclass[journal,12pt,twocolumn]{IEEEtran}

\usepackage{setspace}
\usepackage{gensymb}
\usepackage[cmex10]{amsmath}
\usepackage{amsthm}
\usepackage{amssymb}

\usepackage{mathrsfs}
\usepackage{txfonts}
\usepackage{stfloats}
\usepackage{bm}
\usepackage{cite}
\usepackage{cases}
\usepackage{subfig}

\usepackage{longtable}
\usepackage{multirow}

\usepackage{enumitem}
\usepackage{mathtools}
\usepackage{steinmetz}
\usepackage{tikz}
\usepackage{circuitikz}
\usepackage{verbatim}
\usepackage{tfrupee}
\usepackage[breaklinks=true]{hyperref}
\usepackage{graphicx}
\usepackage{tkz-euclide}
\usepackage{afterpage}
\usetikzlibrary{calc,math}
\usepackage{listings}
    \usepackage{color}                                            
    \usepackage{array}                                            
    \usepackage{longtable}                                       
    \usepackage{calc}                                             
    \usepackage{multirow}                                         
    \usepackage{hhline}                                       
    \usepackage{ifthen}                                           
    \usepackage{lscape}     
\usepackage{multicol}
\usepackage{chngcntr}

\DeclareMathOperator*{\Res}{Res}

\renewcommand\thesection{\arabic{section}}
\renewcommand\thesubsection{\thesection.\arabic{subsection}}
\renewcommand\thesubsubsection{\thesubsection.\arabic{subsubsection}}

\renewcommand\thesectiondis{\arabic{section}}
\renewcommand\thesubsectiondis{\thesectiondis.\arabic{subsection}}
\renewcommand\thesubsubsectiondis{\thesubsectiondis.\arabic{subsubsection}}


\hyphenation{op-tical net-works semi-conduc-tor}
\def\inputGnumericTable{}                                 

\lstset{
%language=C,
frame=single, 
breaklines=true,
columns=fullflexible
}
\begin{document}


\newtheorem{theorem}{Theorem}[section]
\newtheorem{problem}{Problem}
\newtheorem{proposition}{Proposition}[section]
\newtheorem{lemma}{Lemma}[section]
\newtheorem{corollary}[theorem]{Corollary}
\newtheorem{example}{Example}[section]
\newtheorem{definition}[problem]{Definition}

\newcommand{\BEQA}{\begin{eqnarray}}
\newcommand{\EEQA}{\end{eqnarray}}
\newcommand{\define}{\stackrel{\triangle}{=}}
\bibliographystyle{IEEEtran}
\providecommand{\mbf}{\mathbf}
\providecommand{\pr}[1]{\ensuremath{\Pr\left(#1\right)}}
\providecommand{\qfunc}[1]{\ensuremath{Q\left(#1\right)}}
\providecommand{\sbrak}[1]{\ensuremath{{}\left[#1\right]}}
\providecommand{\lsbrak}[1]{\ensuremath{{}\left[#1\right.}}
\providecommand{\rsbrak}[1]{\ensuremath{{}\left.#1\right]}}
\providecommand{\brak}[1]{\ensuremath{\left(#1\right)}}
\providecommand{\lbrak}[1]{\ensuremath{\left(#1\right.}}
\providecommand{\rbrak}[1]{\ensuremath{\left.#1\right)}}
\providecommand{\cbrak}[1]{\ensuremath{\left\{#1\right\}}}
\providecommand{\lcbrak}[1]{\ensuremath{\left\{#1\right.}}
\providecommand{\rcbrak}[1]{\ensuremath{\left.#1\right\}}}
\theoremstyle{remark}
\newtheorem{rem}{Remark}
\newcommand{\sgn}{\mathop{\mathrm{sgn}}}
\providecommand{\abs}[1]{\left\vert#1\right\vert}
\providecommand{\res}[1]{\Res\displaylimits_{#1}} 
\providecommand{\norm}[1]{\left\lVert#1\right\rVert}
%\providecommand{\norm}[1]{\lVert#1\rVert}
\providecommand{\mtx}[1]{\mathbf{#1}}
\providecommand{\mean}[1]{E\left[ #1 \right]}
\providecommand{\fourier}{\overset{\mathcal{F}}{ \rightleftharpoons}}
%\providecommand{\hilbert}{\overset{\mathcal{H}}{ \rightleftharpoons}}
\providecommand{\system}{\overset{\mathcal{H}}{ \longleftrightarrow}}
	%\newcommand{\solution}[2]{\textbf{Solution:}{#1}}
\newcommand{\solution}{\noindent \textbf{Solution: }}
\newcommand{\cosec}{\,\text{cosec}\,}
\providecommand{\dec}[2]{\ensuremath{\overset{#1}{\underset{#2}{\gtrless}}}}
\newcommand{\myvec}[1]{\ensuremath{\begin{pmatrix}#1\end{pmatrix}}}
\newcommand{\mydet}[1]{\ensuremath{\begin{vmatrix}#1\end{vmatrix}}}
\numberwithin{equation}{subsection}
\makeatletter
\@addtoreset{figure}{problem}
\makeatother
\let\StandardTheFigure\thefigure
\let\vec\mathbf
\renewcommand{\thefigure}{\theproblem}
\def\putbox#1#2#3{\makebox[0in][l]{\makebox[#1][l]{}\raisebox{\baselineskip}[0in][0in]{\raisebox{#2}[0in][0in]{#3}}}}
     \def\rightbox#1{\makebox[0in][r]{#1}}
     \def\centbox#1{\makebox[0in]{#1}}
     \def\topbox#1{\raisebox{-\baselineskip}[0in][0in]{#1}}
     \def\midbox#1{\raisebox{-0.5\baselineskip}[0in][0in]{#1}}
\vspace{3cm}
\title{Matrix Theory EE5609 - Assignment 11\\
}

\author{\IEEEauthorblockN{Sandhya Addetla}\\
\IEEEauthorblockA{PhD Artificial Inteligence Department} \\
AI20RESCH14001\\
 }

\maketitle
\begin{abstract}
Minimum Polynomial
\end{abstract}

\section{Problem}
Let A $\in M_3 (\mathbb{R}) $ be such that $A^8 = I_{3 \times 3}$ Then
\begin{enumerate}
\item minimal polynomial of $A$ can only be of degree 2.
\item minimal polynomial of $A$ can only be of degree 3.
\item either $A = I_{3 \times 3}$ or $A = -I_{ 3 \times 3}$
\item there are uncountably many  $A$ satisfying the above.
\end{enumerate}
\section{Solution}
\begin{table}[h!]
\begin{center}
\begin{tabular}{ | m{3cm} | m{5cm}| } \hline 
%\begin{tabular}{|c|c|}\hline
Given  &   A $\in M_3 (\mathbb{R}) $ be such that $A^8 = I_{3 \times 3}$.\\  
  \hline
Option 1 :  minimal polynomial of $A$ can only be of degree 2 & Let {\begin{align*}
 A = \myvec{1&0&0\\0&1&0\\0&0&1}
\end{align*}}\\&
The Characteristic polynomial is $ -\lambda^3+3\lambda^2-3\lambda+1=-(\lambda-1)^3$\\&
Minimum polynomial is of degree 1. \\&Hence this option is not correct\\  \hline
Option 2 :  minimal polynomial of $A$ can only be of degree 3 & Let {\begin{align*}
 A = \myvec{1&0&0\\0&1&0\\0&0&1}
\end{align*}}\\&
as given in option 1, the minimum polynomial is of degree 1.\\& Hence this option is not correct\\  \hline
Option 3 :  either $A = I_{3 \times 3}$ or $A = -I_{ 3 \times 3}$ & Let {\begin{align*}
 A = \myvec{1&0&0\\0&-1&0\\0&0&-1}
\end{align*}}\\&
Here, $A^8 = I_{3 \times 3}$ and $ A \neq I_{3 \times 3}$ or $A \neq -I_{ 3 \times 3}$. Hence this option is not correct\\  \hline
\end{tabular}
\end{center}
\end{table}
\begin{table}[h!]
\begin{center}
\begin{tabular}{ | m{6cm} | m{12cm}| } \hline 
Option 4 :  there are uncountably many  $A$ satisfying the above & Let {\begin{align*}
 \text{M}(\theta)\text{ be } 2 \times 2 \text{  orthogonal matrix} \\
M(\theta)= \myvec{\cos \theta& \sin \theta\\-\sin\theta&\cos \theta}\\
\text{ Let    } P = \myvec{ 1 & 0 \\ 0 & -1 }\\
\text{ then, }\\
M^{-1}(\theta) = M^T(\theta) = \myvec{ \cos \theta & -\sin \theta \\ \sin \theta & \cos \theta }\\
M^T(\theta) P M(\theta) =\myvec{ \cos \theta & -\sin \theta \\ \sin \theta & \cos \theta} \myvec{1 & 0 \\ 0 & -1 } \myvec{ \cos \theta & \sin \theta \\ -\sin \theta & \cos \theta }\\
= \myvec{ \cos \theta & \sin \theta \\ \sin \theta & -\cos \theta}\myvec{\cos \theta & \sin \theta \\ -\sin \theta & \cos \theta } \\= \myvec{\cos^2 \theta - \sin^2 \theta & 2 \cos \theta \sin \theta \\ 2 \cos \theta \sin \theta & \sin^2 \theta - \cos^2 \theta }\\
= \myvec{\cos 2 \theta & \sin 2 \theta \\ \sin 2\theta & -\cos 2 \theta }
\end{align*}}
This is clearly an uncountable family of $2 \times 2$ matrices.\\&
{\begin{align*}
(M^T(\theta) P M(\theta))^2 = M^T(\theta) P M(\theta)  M^T(\theta) P M(\theta)\\ 
= M^T(\theta)P^2 M(\theta) = M^T(\theta)I M(\theta) = I
\end{align*}}
Also,
{\begin{align*}
(M^T(\theta) P M(\theta))^8 = I\\
\text{Let } A(\theta) = \myvec{M^T(\theta) P M(\theta) & 0 \\ 0 & -1 }\\
A^8(\theta) = \myvec{ (O^T(\theta) P O(\theta))^8 & 0 \\ 0 & (-1)^8} = \myvec{I & 0 \\ 0 & 1 }= I\\
\end{align*}}
and the family $\{ A(\theta)\mid \theta \in \Bbb R \}$ is uncountable. Hence there are uncountably many $A$ which satify the above condition
Hence, this option is the correct answer.
\\  \hline

\end{tabular}
\end{center}
\end{table}
\end{document}